%	Autor:		Giuliano Montorio & Hannes Dittmann
%	Datum:		28.08.2018
%	Short:		Entwicklungsprojekt
% DOCUMENTCLASS -----------------------------------------------------------------------------------------
\documentclass[12pt,a4paper,oneside,numbers=noenddot,captions=tableheading,toc=bibliography,openany,tikz,margin=5mm]{scrbook}
\setcounter{secnumdepth}{5}
\renewcommand{\arraystretch}{1.4} 

\usepackage[scaled=.90]{helvet}
\usepackage{geometry}
\geometry{a4paper,left=25mm,right=25mm, top=25mm, bottom=25mm, includeheadfoot}


% SPRACHE------------------------------------------------------------------------------------------------
\usepackage[utf8]{inputenc}
\usepackage[autostyle,babel,german=guillemets,style=german]{csquotes}
\usepackage[USenglish,ngerman]{babel} 
\selectlanguage{ngerman}


% TABELLEN-----------------------------------------------------------------------------------------------
\usepackage{multirow} % Tabellen-Zellen �ber mehrere Zeilen1
\usepackage{multicol} % mehre Spalten auf eine Seite
\usepackage{tabularx} % F�r Tabellen mit vorgegeben Gr��en
\usepackage{array}
\usepackage{float}
\usepackage{booktabs}
\usepackage{longtable} % Tabellen �ber mehrere Seiten
\usepackage{rotating}
\usepackage[table,xcdraw]{xcolor}
\usepackage{subcaption}
\usepackage{tikz}
\usetikzlibrary{circuits.ee.IEC}
\usepackage{pgfplots}
\pgfplotsset{
	dirac/.style={
		mark=triangle*,
		mark options={scale=2},
		ycomb,
		scatter,
		visualization depends on={y/abs(y)-1 \as \sign},
		scatter/@pre marker code/.code={\scope[rotate=90*\sign,yshift=-2pt]}
	}
}
\usetikzlibrary{arrows.meta,positioning,calc}
\usepackage{pdflscape}

% BILDER-------------------------------------------------------------------------------------------------
\usepackage{graphicx}

\usepackage{color}
\usepackage{svg}



% ALLGEMEINES--------------------------------------------------------------------------------------------
\usepackage{amsmath,amssymb} % Mathesachen
\usepackage[T1]{fontenc} % Ligaturen, richtige Umlaute im PDF
\usepackage[hyphens,obeyspaces,spaces]{url}		% bricht lange URLs ?sch�n? um
\usepackage[decimalsymbol=comma]{siunitx}
\usepackage[onehalfspacing]{setspace}
\setlength{\parindent}{0cm}	% Disable automatic parindent
\usepackage{placeins}
\sloppy
\usepackage{mathrsfs}
\usepackage{pdfpages}
\usepackage{enumitem}
\graphicspath{{_img/}}
% PDF-----------------------------------------------------------------------------------------------------
\usepackage{pdfpages}
\usepackage[ngerman,%
pdfauthor={Giuliano Montorio und Hannes Dittmann},%
pdftitle={Bachelorarbeit Dittmann Montorio},%
colorlinks=true,linkcolor=black,citecolor=black,filecolor=black,urlcolor=black%
]{hyperref}

% ABKÜRZUNGEN --------------------------------------------------------------------------------------------
\usepackage[nohyperlinks]{acronym}


% BIBLATEX -----------------------------------------------------------------------------------------------
\usepackage[backend=bibtex,%
style=IEEEtran,%
bibstyle=numeric,%
citestyle=numeric,%
sorting=none,%
]{biblatex}
\bibliography{_bib/biblio}

% QUELLCODE -------------------------------------------------------------------
\usepackage{listings}
\definecolor{mygreen}{RGB}{28,172,0} 		% color values Red, Green, Blue
\definecolor{mylilas}{RGB}{170,55,241}
\lstset{language=Matlab,%
	basicstyle=\footnotesize\ttfamily,
	breaklines=true,%
	morekeywords={matlab2tikz},
	keywordstyle=\color{blue},%
	morekeywords=[2]{1}, keywordstyle=[2]{\color{black}},
	identifierstyle=\color{black},%
	stringstyle=\color{mylilas},
	commentstyle=\color{mygreen},%
	showstringspaces=false,%without this there will be a symbol in the places where there is a space
	numbers=left,%
	numberstyle={\tiny \color{black}},% size of the numbers
	numbersep=9pt, % this defines how far the numbers are from the text
	emph=[1]{for,end,break},emphstyle=[1]\color{red}, %some words to emphasise
	%emph=[2]{word1,word2}, emphstyle=[2]{style},    
}		
% DOCUMENT ------------------------------------------------------------------------------------------------
\begin{document}
	\begin{longtable}{|p{1,5cm}|p{3,5cm}|p{7cm}|p{2,2cm}|} 
	\hline
	\textbf{Nr./ID} & \textbf{Titel} & \textbf{Verifikation der Anforderung} & \textbf{Hilfsmittel}\\
	\endhead


	\hline

ANF\_$01$ & Sprachaufnahme per manueller Betätigung & Zur Durchführung sind folgende Arbeitsschritte durchzuführen:
\begin{itemize}
	
	\item[1.]	ALF einschalten (Rechner hochfahren und Wahlschlüsselschalter auf „Hand“-Modus stellen)
	\item[2.]	Launch-File starten (./launch.sh) und bis zur Initialisierung warten
	\item[3.]	Abonnieren der veröffentlichten \textit{ROS-Topic} \textit{/audioStream/pub/stream/topic}
	\item[4.]	\textit{Start} Knopf der Fernbedienung betätigen
	\item[5.]	In dem ROS log wird \textit{Start streaming audio...} geschrieben
	\item[6.]	Nach der Aufnahme, wird in den ROS log \textit{Finish streaming audio...} geschrieben
	\item[7.]	Abonnement der \textit{ROS-Topic} erhält eine neue Nachricht
	
	
\end{itemize}

\underline{Ergebnis:}\newline
\newline

\textbf{}	%hier kommt bestanden oder so rein
& \textit{ROS}\newline
\textit{Kinect-Mikrofon}\newline
\textit{rostopic echo topicname}\newline
\\
\hline

ANF\_$02$ & Erzeugen und bereitstellen einer Tonspur & Zur Durchführung sind folgende Arbeitsschritte durchzuführen:
\begin{itemize}
	
	\item[1.]	ALF einschalten (Rechner hochfahren und Wahlschlüsselschalter auf „Hand“-Modus stellen)
	\item[2.]	Launch-File starten (./launch.sh) und bis zur Initialisierung warten
	\item[3.]	Abonnieren der veröffentlichten \textit{ROS-Topic} \textit{/audioStream/pub/stream/topic}
	\item[4.]	\textit{Start} Knopf der Fernbedienung betätigen
	\item[5.]	Über die vorgegebene Zeitspanne von $t=5$ s in das Mikrofon eine bedienungsorientierte Wortgruppe sprechen
	\item[6.]	Abonnement der \textit{ROS-Topic} erhält eine neue Nachricht, diese wird zu einer \textit{WAV}-Datei \textit{test.wav} konvertiert und liegt im Ordner des Knotens
	\item[7.]	Abspielen der Audio Datei mit entsprechender Software
\end{itemize}

\underline{Ergebnis:}\newline
\newline

\textbf{}	%hier kommt bestanden oder so rein
& \textit{ROS}\newline
\textit{Kinect-Mikrofon}\newline
\textit{rostopic echo topicname}\newline

\\
\hline

ANF\_03 & Erkennung und Klassifizierung von bedienungsorientierter Sprache des Benutzers & Zur Durchführung sind folgende Arbeitsschritte durchzuführen:
\begin{itemize}
	
	\item[1.]	ALF einschalten (Rechner hochfahren und Wahlschlüsselschalter auf „Hand“-Modus stellen)
	\item[2.]	Launch-File starten (./launch.sh) und bis zur Initialisierung warten
	\item[3.]	Abonnieren der veröffentlichten \textit{ROS-Topic} \textit{recognizer/topicname/task}
	\item[4.]	\textit{Start} Knopf der Fernbedienung betätigen und über die vorgegebene Zeitspanne in das Mikrofon eine bedienungsorientierte Wortgruppe sprechen
	\item[5.]	Vergleich der veröffentlichten Klassifikation mit der zugehörigen Kategorie der Wortgruppe aus dem Datensatz \textit{random-distributed-dataset.json}
	
\end{itemize}

\underline{Ergebnis:}\newline
\newline

\textbf{}	%hier kommt bestanden oder so rein
& \textit{ROS}\newline
\textit{Kinect-Mikrofon}\newline
\textit{Datensatz}\newline
\\
\hline

ANF\_04 & 	Erkennen von benutzerdefinierten Schlagwörtern & Zur Durchführung sind folgende Arbeitsschritte durchzuführen:
\begin{itemize}
	
	\item[1.]	ALF einschalten (Rechner hochfahren und Wahlschlüsselschalter auf „Hand“-Modus stellen)
	\item[2.]	Launch-File starten (./launch.sh) und bis zur Initialisierung warten
	\item[3.]	Abonnieren der veröffentlichten \textit{ROS-Topic} \textit{recognizer/topicname/buzz}
	\item[4.]	\textit{Start} Knopf der Fernbedienung betätigen und über die vorgegebene Zeitspanne in das Mikrofon ein Schlagwort der Liste \textit{buzzword.json} sprechen
	\item[5.]	Vergleich des veröffentlichten Schlagworts mit dem eingesprochenen
	
\end{itemize}

\underline{Ergebnis:}\newline
\newline

\textbf{}	%hier kommt bestanden oder so rein
& \textit{ROS}\newline
\textit{Kinect-Mikrofon}\newline
\\
	\hline
ANF\_05 & Erhöhung der Stufe für autonomes Fahren & Während der Ausführung autonomer Fahraufgaben müssen die Kriterien der Tabelle aus Abbildung 3 des Lastenhefts für das Level 5 erfüllt werden. Zur Durchführung sind folgende Arbeitsschritte durchzuführen:
\begin{itemize}
	
	\item[1.]	ALF einschalten (Rechner hochfahren und Wahlschlüsselschalter auf „Hand“-Modus stellen)
	\item[2.]	Zustandsautomat starten (python2 finitestatemachine.py) und bis zur Initialisierung warten
	\item[3.]	Autonome Fahrmodi nach der manuellen Betätigung der Spracheingabe mit bedienungsorientierter, englischer Sprache aufrufen. Zum Beispiel: "Drive to location."
	\item[4.]	Anweisungen des ALFs befolgen.
	\item[5.]	Vergleich der Kriterien nach der genannten Tabelle mit dem Verhalten des ALFs.
	
\end{itemize}
\underline{Ergebnis:}\newline

\textbf{} %hier kommt bestanden oder so rein
& 	Checkliste
\\
	\hline

ANF\_06 & Erkennen und Unterscheiden von Personen in Reichweite der vorgesehenen Sensorik & Zur Durchführung sind folgende Arbeitsschritte durchzuführen:
\begin{itemize}
	
	\item[1.]	ALF einschalten (Rechner hochfahren und Wahlschlüsselschalter auf „Hand“-Modus stellen).
	\item[2.]	Launch-File starten (People2Pose).
	\item[3.]	Benutzer I stellt sich in das Sichtfeld einer Kamera.
	\item[4.]	Benutzer I verlässt das Sichtfeld und Benutzer II betritt dieses.
	\item[5.]	Benutzer I und betritt das Sichtfeld und wird wiedererkannt.
	\item[6.]	Abonnieren der veröffentlichten \textit{ROS-Topic} und Vergleichen der Daten
	
\end{itemize}

\underline{Ergebnis:}\newline
\newline

\textbf{}	%hier kommt bestanden oder so rein
& \textit{ROS}\newline
\textit{Kinect-Kameras}
\\
\hline

ANF\_07 & Wiedererkennung von Personen in Reichweite der vorgesehenen Sensorik nach einer definierten Zeit & Zur Durchführung sind folgende Arbeitsschritte durchzuführen:
\begin{itemize}
	\item[1.]	ALF einschalten (Rechner hochfahren und Wahlschlüsselschalter auf „Hand“-Modus stellen).,
	\item[2.]	Launch-File starten (People2Pose).
	\item[3.]	Widerholen der Schritte drei bis fünf aus ANF\_08.
	\item[4.]	Abonnieren der Veröffentlichten \textit{ROS-Topic}. 
	\item[5.]	Messung der X- und Y-Komponente des Benutzers mit vorgeschriebenen Messmittel.
	\item[6.]	Posenschätzung mit Messung vergleichen. Bezugspunkt ist das Kamerakoordinatensystem.
	\item[7.]	Neustart des Roboters und wiederausführung des Programms.
	\item[8.]	Benutzer I und betritt das Sichtfeld und wird wiedererkannt.
	\item[9.]	Messung der X- und Y-Komponente des Benutzers mit vorgeschriebenen Messmittel.
	\item[10.]	Posenschätzung mit Messung vergleichen. Bezugspunkt ist das Kamerakoordinatensystem.
	
\end{itemize}

\underline{Ergebnis:}\newline
\newline

\textbf{}%bestanden?
& \textit{ROS}\newline
Bandmaß\newline
\textit{Kinect-Kameras}
\\
	\hline
	ANF\_08 & Wiedererkennung von Personen in Reichweite der vorgesehenen Sensorik innerhalb einer vorgegebenen Zeit & Zur Durchführung sind folgende Arbeitsschritte durchzuführen:
	\begin{itemize}
		
		
		\item[1.]	ALF einschalten (Rechner hochfahren und Wahlschlüsselschalter auf „Hand“-Modus stellen)
		\item[2.]	Launch-File starten (\textit{SLAM})
		\item[3.]	Resetknopf betätigen
		\item[4.]	Bewegungsvorgabe durch den Benutzer mithilfe des Joysticks oder einer Zielvorgabe in \textit{Rviz}.
		\item[5.]	Am Joystick die Tasten \glqq left trigger\grqq{} und \glqq right trigger\grqq{} drücken
		\item[6.]	Auswahl eines Referenzobjekts.
		\item[7.]	Bestimmung der Maße des Referenzobjekts.
		\item[8.]	Referenzobjekt in aufgenommener statischer Karte finden und Maße unter Einbeziehung des Kartenmaßstabs und der Auflösung bestimmen.
		\item[9.]	Vergleich der in den vorherigen Schritten bestimmten Maße.
		
		
		
	\end{itemize}
	
	\underline{Ergebnis:}\newline
   \newline
	
	\textbf{}%hier kommt bestanden oder so rein
	
	& \textit{Rviz \newline Joystick}\\
	\hline
	ANF\_09 & Registrierung von Personen in Reichweite der vorgesehenen Sensorik innerhalb einer vorgegebenen Zeit& Zur Durchführung sind folgende Arbeitsschritte durchzuführen:
	\begin{itemize}
		
		
		\item[1.]	ALF einschalten (Rechner hochfahren und Wahlschlüsselschalter auf „Hand“-Modus stellen)
		\item[2.]	Launch-File starten (\textit{SLAM + Explore lite})
		\item[3.]	Resetknopf drücken
		\item[4.]	Auswahl eines Referenzobjekts.
		\item[5.]	Bestimmung der Maße des Referenzobjekts.
		\item[6.]	Referenzobjekt in aufgenommener statischer Karte finden und Maße unter Einbeziehung des Kartenmaßstabs und der Auflösung bestimmen.
		\item[7.]	Vergleich der in den vorherigen Schritten bestimmten Maße.

		
	\end{itemize}
	
	\underline{Ergebnis:}\newline
	\newline
	
	\textbf{}%hier kommt bestanden oder so rein
	
	& \textit{Rviz \newline Joystick}\\

	\hline
	ANF\_10 & Positionsschätzung von erkannten Personen & Zur Durchführung sind folgende Arbeitsschritte durchzuführen:
	\begin{itemize}
		
		
		\item[1.]	ALF einschalten (Rechner hochfahren und Wahlschlüsselschalter auf „Hand“-Modus stellen)
		\item[2.]	Launch-File starten (Lokalisierung)
		\item[3.]	Bezugspunkt in der Umgebung festlegen
		\item[4.]	Bezugspunkt in statischer Karte in \textit{Rviz} eintragen (z.B. als \textit{Simple Goal})
		\item[5.]	Aktuelle Posenschätzung aus dem \textit{ROS}-Netzwerk auslesen
		\item[6.]	Transformation zwischen Bezugspunkt und Posenschätzung bestimmen
		\item[7.]	Messung mit vorgeschriebenen Messmittel
		\item[8.]	Vergleich der Transformation mit dem gemessenen Werten
		
	\end{itemize}
	
	\underline{Ergebnis:}\newline
	\newline
	
	\textbf{}%hier kommt bestanden rein
	& \textit{Rviz}\newline
	Bandmaß\newline
	Kompass
	\\
	\hline
	ANF\_11 & Kartographierung der Umgebung mit Bewegungsvorgabe durch den Benutzer & Zur Durchführung sind folgende Arbeitsschritte durchzuführen:
		\begin{itemize}
			
			
			\item[1.]	ALF einschalten (Rechner hochfahren und Wahlschlüsselschalter auf „Hand“-Modus stellen)
			\item[2.]	Zustandsautomat durch \textit{python2 finitestatemachine.py} ausführen
			\item[3.]	Ausführen \textit{CANStartUp.sh}
			\item[4.]	Resetknopf betätigen
			\item[5.]	Manuell oder durch Sprachbefehl in den Modus \textit{SLAM/Manuell} wechseln
			\item[6.]	Eingabe von Fahrbefehlen mithilfe der Fernbedienung

			
		\end{itemize}
	
	\underline{Ergebnis:}\newline
	\newline
	
	\textbf{}\textbf{}	%hier kommt bestanden oder so rein
	& \textit{Rviz}\newline
	\textit{Joystick}\newline
	\textit{Kinect-Kameras}
	\\


	\hline

	ANF\_12& Kartographieren der Umgebung ohne Bewegungsvorgabe durch den Benutzer & Zur Durchführung sind folgende Arbeitsschritte durchzuführen:
	\begin{itemize}
			\item[1.]	ALF einschalten (Rechner hochfahren und Wahlschlüsselschalter auf „Hand“-Modus stellen)
		\item[2.]	Zustandsautomat durch \textit{python2 finitestatemachine.py} ausführen
		\item[3.]	Ausführen \textit{CANStartUp.sh}
		\item[4.]	Resetknopf betätigen
		\item[5.]	Manuell oder durch Sprachbefehl in den Modus \textit{SLAM/Autonom} wechseln
		\item[6.]	Eingabe von Fahrbefehlen mithilfe der Fernbedienung
	\end{itemize}
	
	\underline{Ergebnis:}\newline
	\newline
	\textbf{}%bestanden?
	& \textit{ROS}\newline
	Bandmaß\newline
	\textit{Kinect-Kameras}
	\\
	\hline
	ANF\_13&Posenschätzung in vorhandener statischer Karte & Zur Durchführung sind folgende Arbeitsschritte durchzuführen:
	\begin{itemize}
		\item[1.]	ALF einschalten (Rechner hochfahren und Wahlschlüsselschalter auf „Hand“-Modus stellen)
		\item[2.]	Zustandsautomat durch \textit{python2 finitestatemachine.py} ausführen
		\item[3.]	Ausführen \textit{CANStartUp.sh}
		\item[4.]	Resetknopf betätigen
		\item[5.]	Manuell oder durch Sprachbefehl in den Modus \textit{\textit{Lokalisierung}} wechseln
		\item[6.]	Aktuelle Posenschätzung aus dem \textit{ROS}-Netzwerk auslesen und abspeichern.
		\item[7.]	Fahrzeug um definierte und bekannte Maße verschieben.
		\item[8.]	Aktuelle Posenschätzung aus dem \textit{ROS}-Netzwerk auslesen und abspeichern. Differenz der zweiten und ersten Schätzung ergibt die Verschiebung aus Punkt 9. Position in der Karte mithilfe von Referenzobjekten aus der vorhandenen statischen Karte und der realen Umgebung auf Plausibilität prüfen.
	\end{itemize}
	
	\underline{Ergebnis:}\newline
	\newline
	\textbf{}%bestanden?
	& \textit{ROS}\newline
	Bandmaß\newline
	\textit{Kinect-Kameras}\newline
	\textit{LiDar}
	\\
	\hline
	ANF\_14& Anfahren einer vom Benutzer vorgegebenen Zielpose & Zur Durchführung sind folgende Arbeitsschritte durchzuführen:
	\begin{itemize}
	\item[1.]	ALF einschalten (Rechner hochfahren und Wahlschlüsselschalter auf „Hand“-Modus stellen)
	\item[2.]	Zustandsautomat durch \textit{python2 finitestatemachine.py} ausführen.
	\item[3.]	Ausführen \textit{CANStartUp.sh}
	\item[4.]	Resetknopf betätigen.
	\item[5.]	Manuell oder durch Sprachbefehl in den Modus \textit{Drive/Autonom/to target} wechseln
	\item[6.]	Eingabe eines \textit{Simple Goal} in \textit{Rviz} oder durch einen Sprachbefehl
	\item[7.]	Durchführen der Fahraufgabe
	\item[8.]	\textit{Simple Goal} und Posenschätzung mithilfe der Veröffentlichungen aus dem \textit{ROS}-Netzwerk vergleichen.
\end{itemize}

	
	\underline{Ergebnis:}\newline
	\newline
	\textbf{}%bestanden?
	& \textit{ROS}\newline
\textit{Rviz}\newline
	\\
	\hline
	ANF\_15& Autonomes Fahren durch enge Passagen & Zur Durchführung sind folgende Arbeitsschritte durchzuführen:
	\begin{itemize}
		\item[1.]	ALF einschalten (Rechner hochfahren und Wahlschlüsselschalter auf „Hand“-Modus stellen)
		\item[2.]	Zustandsautomat durch \textit{python2 finitestatemachine.py} ausführen.
		\item[3.]	Ausführen \textit{CANStartUp.sh}
		\item[4.]	Resetknopf betätigen.
		\item[5.]	Manuell oder durch Sprachbefehl in den Modus \textit{Drive/Autonom/to target} wechseln
		\item[6.]	\textit{Simple Goal} durch \textit{Rviz} im \textit{ROS}-Netzwerk veröffentlichen und sicherstellen das die berechnete Trajektorie durch eine enge Passage im Sinne der Anforderung führt
		\item[7.]	Vollautomatisiertes durchführen der Fahraufgabe
		\item[8.]	Aktuelle Posenschätzung aus dem \textit{ROS}-Netzwerk auslesen und mit dem veröffentlichten \textit{Simple Goal} vergleichen.
	\end{itemize}
	
	\underline{Ergebnis:}\newline
	\newline
	\textbf{}%bestanden?
	& \textit{ROS}\newline
	\\
		ANF\_16& Risikominimaler Zustand & Zur Durchführung sind folgende Arbeitsschritte durchzuführen:
	\begin{itemize}
		\item[1.]	ALF einschalten (Rechner hochfahren und Wahlschlüsselschalter auf „Hand“-Modus stellen)
		\item[2.]	Zustandsautomat durch \textit{python2 finitestatemachine.py} ausführen.
		\item[3.]	Ausführen \textit{CANStartUp.sh}
		\item[4.]	Resetknopf betätigen.
		\item[5.]	Manuell oder durch Sprachbefehl in den Modus \textit{Stop} wechseln
		\item[6.]	Betätigung aller Joystick Schalter,
		\item[7.]   Einsprechen eines Sprachbefehls, 
		\item[8.]	Überprüfen ob das ROS-Netzwerk offline ist
		\item[8.]	Nach der Quittierung setzt das Fahrzeug keinen Fahrbefehl um
	\end{itemize}
	
	\underline{Ergebnis:}\newline
	\newline
	\textbf{}%bestanden?
	& \textit{ROS}\newline
	\textit{CAN-sniffer}
	\\
	\hline
\end{longtable} 
\end{document}