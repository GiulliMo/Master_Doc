%	Autor:		Giuliano Montorio & Hannes Dittmann
%	Datum:		28.08.2018
%	Short:		Entwicklungsprojekt
% DOCUMENTCLASS -----------------------------------------------------------------------------------------
\documentclass[12pt,a4paper,oneside,numbers=noenddot,captions=tableheading,toc=bibliography,openany,tikz,margin=5mm]{scrbook}
\setcounter{secnumdepth}{5}
\renewcommand{\arraystretch}{1.4} 

\usepackage[scaled=.90]{helvet}
\usepackage{geometry}
\geometry{a4paper,left=25mm,right=25mm, top=25mm, bottom=25mm, includeheadfoot}


% SPRACHE------------------------------------------------------------------------------------------------
\usepackage[utf8]{inputenc}
\usepackage[autostyle,babel,german=guillemets,style=german]{csquotes}
\usepackage[USenglish,ngerman]{babel} 
\selectlanguage{ngerman}


% TABELLEN-----------------------------------------------------------------------------------------------
\usepackage{multirow} % Tabellen-Zellen �ber mehrere Zeilen1
\usepackage{multicol} % mehre Spalten auf eine Seite
\usepackage{tabularx} % F�r Tabellen mit vorgegeben Gr��en
\usepackage{array}
\usepackage{float}
\usepackage{booktabs}
\usepackage{longtable} % Tabellen �ber mehrere Seiten
\usepackage{rotating}
\usepackage[table,xcdraw]{xcolor}
\usepackage{subcaption}
\usepackage{tikz}
\usetikzlibrary{circuits.ee.IEC}
\usepackage{pgfplots}
\pgfplotsset{
	dirac/.style={
		mark=triangle*,
		mark options={scale=2},
		ycomb,
		scatter,
		visualization depends on={y/abs(y)-1 \as \sign},
		scatter/@pre marker code/.code={\scope[rotate=90*\sign,yshift=-2pt]}
	}
}
\usetikzlibrary{arrows.meta,positioning,calc}
\usepackage{pdflscape}

% BILDER-------------------------------------------------------------------------------------------------
\usepackage{graphicx}

\usepackage{color}
\usepackage{svg}



% ALLGEMEINES--------------------------------------------------------------------------------------------
\usepackage{amsmath,amssymb} % Mathesachen
\usepackage[T1]{fontenc} % Ligaturen, richtige Umlaute im PDF
\usepackage[hyphens,obeyspaces,spaces]{url}		% bricht lange URLs ?sch�n? um
\usepackage[decimalsymbol=comma]{siunitx}
\usepackage[onehalfspacing]{setspace}
\setlength{\parindent}{0cm}	% Disable automatic parindent
\usepackage{placeins}
\sloppy
\usepackage{mathrsfs}
\usepackage{pdfpages}
\usepackage{enumitem}
\graphicspath{{_img/}}
% PDF-----------------------------------------------------------------------------------------------------
\usepackage{pdfpages}
\usepackage[ngerman,%
pdfauthor={Giuliano Montorio und Hannes Dittmann},%
pdftitle={Bachelorarbeit Dittmann Montorio},%
colorlinks=true,linkcolor=black,citecolor=black,filecolor=black,urlcolor=black%
]{hyperref}

% ABKÜRZUNGEN --------------------------------------------------------------------------------------------
\usepackage[nohyperlinks]{acronym}


% BIBLATEX -----------------------------------------------------------------------------------------------
\usepackage[backend=bibtex,%
style=IEEEtran,%
bibstyle=numeric,%
citestyle=numeric,%
sorting=none,%
]{biblatex}
\bibliography{_bib/biblio}

% QUELLCODE -------------------------------------------------------------------
\usepackage{listings}
\definecolor{mygreen}{RGB}{28,172,0} 		% color values Red, Green, Blue
\definecolor{mylilas}{RGB}{170,55,241}
\lstset{language=Matlab,%
	basicstyle=\footnotesize\ttfamily,
	breaklines=true,%
	morekeywords={matlab2tikz},
	keywordstyle=\color{blue},%
	morekeywords=[2]{1}, keywordstyle=[2]{\color{black}},
	identifierstyle=\color{black},%
	stringstyle=\color{mylilas},
	commentstyle=\color{mygreen},%
	showstringspaces=false,%without this there will be a symbol in the places where there is a space
	numbers=left,%
	numberstyle={\tiny \color{black}},% size of the numbers
	numbersep=9pt, % this defines how far the numbers are from the text
	emph=[1]{for,end,break},emphstyle=[1]\color{red}, %some words to emphasise
	%emph=[2]{word1,word2}, emphstyle=[2]{style},    
}		
% DOCUMENT ------------------------------------------------------------------------------------------------
\begin{document}
	\begin{longtable}{|p{1,5cm}|p{3,5cm}|p{7cm}|p{2,2cm}|} 
	\hline
	\textbf{Nr./ID} & \textbf{Titel} & \textbf{Verifikation der Anforderung} & \textbf{Hilfsmittel}\\
	\endhead
	\hline
	ANF\_01 & Posenwinkelbe- \newline stimmung des ALF, durch die vorhandene IMU Sensorik & Zur Durchführung sind folgende Arbeitsschritte durchzuführen:
	\begin{itemize}
		
		
		\item[1.]	ALF einschalten (Rechner hochfahren und Wahlschlüsselschalter auf „Hand“-Modus stellen)
		\item[2.]	Launch-File starten (Manuelles Fahren)
		\item[3.]	Resetknopf betätigen
		\item[4.]	Manuelle Fahraufgaben mit hilfe des Joysticks durchführen
		\item[5.]	Aktuelle Posenwineklschätzung durch IMU-Sensorik aus dem \textit{ROS}-Netzwerk abbonieren
		\item[6.]	Posenwinkelschätzung des verwendeten SLAM-Algorithmus aus dem \textit{ROS}-Netzwerk abbonieren
		\item[7.]	Vergleich der Posenwinkelschätzungen im Simulation Data Inspector

		
	\end{itemize}
	\textbf{}%hier kommt bestanden oder so rein
	
	\underline{Ergebnis:}\newline
	\newline
	
	\textbf{}%hier kommt bestanden oder so rein
	&\textit{ROS}\newline
	\textit{Simulink}\
	Joystick\\ 
	\hline
	ANF\_02 & Kartographierung der Umgebung mit Bewegungsvorgabe durch den Benutzer & Zur Durchführung sind folgende Arbeitsschritte durchzuführen:
	\begin{itemize}
		
		
		\item[1.]	ALF einschalten (Rechner hochfahren und Wahlschlüsselschalter auf „Hand“-Modus stellen)
		\item[2.]	Launch-File starten (\textit{SLAM})
		\item[3.]	Resetknopf betätigen
		\item[4.]	Bewegungsvorgabe durch den Benutzer mithilfe des Joysticks oder einer Zielvorgabe in \textit{Rviz}.
		\item[5.]	Am Joystick die Tasten \glqq left trigger\grqq{} und \glqq right trigger\grqq{} drücken
		
	\end{itemize}
	
	\underline{Ergebnis:}\newline
   \newline
	
	\textbf{}%hier kommt bestanden oder so rein
	
	& \textit{Rviz \newline Joystick}\\
	\hline
	ANF\_03 & Kartographieren der Umgebung ohne Bewegungsvorgabe durch den Benutzer& Zur Durchführung sind folgende Arbeitsschritte durchzuführen:
	\begin{itemize}
		
		
		\item[1.]	ALF einschalten (Rechner hochfahren und Wahlschlüsselschalter auf „Hand“-Modus stellen)
		\item[2.]	Launch-File starten (\textit{SLAM + Explore lite})
		\item[3.]	Resetknopf drücken
		\item[4.]	Am Joystick die Tasten \glqq left trigger\grqq{} und \glqq right trigger\grqq{} drücken
		
	\end{itemize}
	
	\underline{Ergebnis:}\newline
	\newline
	
	\textbf{}%hier kommt bestanden oder so rein
	
	& \textit{Rviz \newline Joystick}\\
	\hline
	ANF\_04 & Erhöhung der Stufe für autonomes Fahren & Während der Ausführung der Anwendungsszenarien 2,3 und 5 müssen die Kriterien des BASt \textit{Fahraufgaben des Fahres nach Automatisierungsgrad} für das Level 4 erfüllt werden. \newline
	
	\underline{Ergebnis:}\newline
	
	\textbf{} %hier kommt bestanden oder so rein
	& 	Checkliste für Kriterien
	\\
	\hline
	ANF\_05 & Posenschätzung in vorhandener statischer Karte & Zur Durchführung sind folgende Arbeitsschritte durchzuführen:
	\begin{itemize}
		
		
		\item[1.]	ALF einschalten (Rechner hochfahren und Wahlschlüsselschalter auf „Hand“-Modus stellen)
		\item[2.]	Launch-File starten (Lokalisierung)
		\item[3.]	Bezugspunkt in der Umgebung festlegen
		\item[4.]	Bezugspunkt in statischer Karte in \textit{Rviz} eintragen (z.B. als \textit{Simple Goal})
		\item[5.]	Aktuelle Posenschätzung aus dem \textit{ROS}-Netzwerk auslesen
		\item[6.]	Transformation zwischen Bezugspunkt und Posenschätzung bestimmen
		\item[7.]	Messung mit vorgeschriebenen Messmittel
		\item[8.]	Vergleich der Transformation mit dem gemessenen Werten
		
	\end{itemize}
	
	\underline{Ergebnis:}\newline
	\newline
	
	\textbf{}%hier kommt bestanden rein
	& \textit{Rviz}\newline
	Bandmaß\newline
	Kompass
	\\
	\hline
	ANF\_06 & Anfahren einer vom benutzer vorgegebenen Zielpose & Zur Durchführung sind folgende Arbeitsschritte durchzuführen:
		\begin{itemize}
			
			
			\item[1.]	ALF einschalten (Rechner hochfahren und Wahlschlüsselschalter auf „Hand“-Modus stellen)
			\item[2.]	Launch-File starten (Lokalisierung oder SLAM)
			\item[3.]	Resetknopf betätigen
			\item[4.]	\textit{Simple Goal} durch \textit{Rviz} im \textit{ROS}-Netzwerk veröffentlichen
			\item[5.]	Aufforderung des Roboterszur Übernahme der Fahraufgabe
			\item[6.]	Quittierung durch Benutzer
			\item[7.]	Vollautomatisiertes durchführen der Fahraufgabe
			\item[8.]	Bestätigung des Roboters, dass die Fahraufgabe durchgeführt wurde
			\item[9.]	Aktuelle Posenschätzung aus dem \textit{ROS}-Netzwerk auslesen und mit dem veröffentlichten \textit{Simple Goal} vergleichen.
			
		\end{itemize}
	
	\underline{Ergebnis:}\newline
	\newline
	
	\textbf{}\textbf{}	%hier kommt bestanden oder so rein
	& \textit{Rviz}\newline
	\textit{Matlab}\newline
	\\
	\hline
	
	ANF\_07 & Erkennung von Sprache des Benutzers & Zur Durchführung sind folgende Arbeitsschritte durchzuführen:
	\begin{itemize}
		
		\item[1.]	ALF einschalten (Rechner hochfahren und Wahlschlüsselschalter auf „Hand“-Modus stellen)
		\item[2.]	Launch-File starten (SpeechRecognition) und Mikrofon auswählen
		\item[3.]	Vorgegebene Zeit in das Mikrofon sprechen
		\item[4.]	Abonnieren der Veröffentlichten \textit{ROS-Topic}
		\item[5.]	Vergleich der Transkription mit Spracheingabe
				
	\end{itemize}
	
	\underline{Ergebnis:}\newline
	\newline
	
	\textbf{}	%hier kommt bestanden oder so rein
	& \textit{ROS}\newline
	\textit{Kinect-Mikrofon}\newline
	\\
	\hline
	
	ANF\_08 & Erkennung von Personen in Reichweite der verwendeten Sensorik & Zur Durchführung sind folgende Arbeitsschritte durchzuführen:
	\begin{itemize}
		
		
		\item[1.]	ALF einschalten (Rechner hochfahren und Wahlschlüsselschalter auf „Hand“-Modus stellen)
		\item[2.]	Launch-File starten (People2Pose)
		\item[3.]	Der Benutzer stellt sich in das Sichtfeld einer Kamera
		\item[4.]	Messung der X- und Y-Komponente des Benutzers mit vorgeschriebenen Messmittel
		\item[5.]	Abonnieren der Veröffentlichten \textit{ROS-Topic} 
		\item[6.]	Posenschätzung mit Messung vergleichen. Bezugspunkt ist das Roboterzentrum

	\end{itemize}
	
	\underline{Ergebnis:}\newline
	\newline
	
	\textbf{}	%hier kommt bestanden oder so rein
	& \textit{ROS}\newline
	Bandmaß\newline
	\textit{Kinect-Kameras}
	\\
	\hline
	
	ANF\_09 & Tracking von erkannten Personen & Zur Durchführung sind folgende Arbeitsschritte durchzuführen:
	\begin{itemize}
		\item[1.]	ALF einschalten (Rechner hochfahren und Wahlschlüsselschalter auf „Hand“-Modus stellen)
		\item[2.]	Launch-File starten (People2Pose)
		\item[3.]	Der Benutzer bewegt sich im Sichtfeld einer Kamera
		\item[4.]Abonnieren der Veröffentlichten \textit{ROS-Topic} , welche die Transkription beinhaltet 
		\item[5.]	Messung der X- und Y-Komponente des Benutzers mit vorgeschriebenen Messmittel
		\item[6.]	Posenschätzung mit Messung vergleichen. Bezugspunkt ist das Roboterzentrum
	\end{itemize}
	
	\underline{Ergebnis:}\newline
	\newline
	
	\textbf{}%bestanden?
	& \textit{ROS}\newline
	Bandmaß\newline
	\textit{Kinect-Kameras}
	\\
	\hline
	
		ANF\_10 & Sprachausgabe an Benutzer & Zur Durchführung sind folgende Arbeitsschritte durchzuführen:
	\begin{itemize}
		\item[1.]	ALF einschalten (Rechner hochfahren und Wahlschlüsselschalter auf „Hand“-Modus stellen)
		\item[2.]	Launch-File starten (??)
		\item[3.]	Eingabe der geforderten Ausgabe, z.B. als String
		\item[4.]	Zuhören und sicherstellen das die Lautsprecher eingeschaltet sind
	\end{itemize}
	
	\underline{Ergebnis:}\newline
	\newline
	
	\textbf{}%bestanden?
	& \textit{ROS}\newline
	Lautsprecher
	\\
	\hline
	
	ANF\_11 & Bedienung durch Sprachbefehle & Zur Durchführung sind folgende Arbeitsschritte durchzuführen:
	\begin{itemize}
		\item[1.]	ALF einschalten (Rechner hochfahren und Wahlschlüsselschalter auf „Hand“-Modus stellen)
		\item[2.]	Launch-File starten (SpeechRecognition)
		\item[3.]	Der Benutzer spricht einen Roboter-Spezifische Wortgruppe in das Mikrofon der Kinect-Kamera 
		\item[4.]	Quittieren einer Handlungsableitung
		\item[5.]	Der Roboter führt die spezifische Handlung aus
	\end{itemize}
	
	\underline{Ergebnis:}\newline
	\newline
	
	\textbf{}%bestanden?
	& \textit{ROS}\newline
	\newline
	\textit{Kinect-Mikrofon}
	\textit{Matlab}
	\\
	\hline
	
	ANF\_12 & Automes Fahren durch enge Passagen & Zur Durchführung sind folgende Arbeitsschritte durchzuführen:
	\begin{itemize}
		\item[1.]	ALF einschalten (Rechner hochfahren und Wahlschlüsselschalter auf „Hand“-Modus stellen)
		\item[2.]	Launch-File starten (Lokalisierung)
		\item[3.]	\textit{Simple Goal} durch \textit{Rviz} im \textit{ROS}-Netzwerk veröffentlichen und sicherstellen das die berechnete Trajektorie durch eine enge Passage im Sinne der Anforderung führt
		\item[4.]	Aufforderung des Roboterszur Übernahme der Fahraufgabe
		\item[5.]	Quittierung durch Benutzer
		\item[6.]	Vollautomatisiertes durchführen der Fahraufgabe
		\item[7.]	Bestätigung des Roboters, dass die Fahraufgabe durchgeführt wurde
		\item[8.]	Aktuelle Posenschätzung aus dem \textit{ROS}-Netzwerk auslesen und mit dem veröffentlichten \textit{Simple Goal} vergleichen.
	\end{itemize}
	
	\underline{Ergebnis:}\newline
	\newline
	\textbf{}%bestanden?
	& \textit{ROS}\newline
	Bandmaß\newline
	\textit{Kinect-Kameras}
	\\
	\hline
\end{longtable} 
\end{document}